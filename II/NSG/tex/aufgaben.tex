\section{Aufgaben} % (fold)
\label{sec:aufgaben}

	Schreiben Sie ein Programm, welches die Navier-Stokes Gleichung für eine inkompressible Flüssigkeit in zwei Dimensionen löst.
	Beschäftigen Sie sich hierfür zunächst mit der Theorie der Navier-Stokes Gleichung und der Finiten-Differenzen Methode. 
	Implementieren Sie anschließend Ihr Programm unter Berücksichtigung der Versuchsanleitung.
	Simulieren und visualisieren Sie mit Ihrem Programm eine Nischenströmung und vergleichen Sie ihre Resultate mit den aus der Literatur bekannten Werten.
	

	\begin{itemize}
		\item Überlegen Sie sich im ersten Schritt die Struktur Ihres Programmes. 
		Es ist sinnvoll, den Quellcode in logische Einheiten zu unterteilen. 

		\item Überlegen Sie sich, welche Parameter Sie für Ihr Programm benötigen werden.
		Diese sollten nicht im Quellcode gesetzt werden, sondern von einer Parameterdatei eingelesen werden.

		\item Überlegen Sie sich, welche Funktionen Sie benötigen, um den Algorithmus in Ihr Programm zu implementieren.
		Welche Übergabe- und Rückgabeparameter haben diese Funktionen? 

		\item Kommentieren Sie Ihren Code für eine übersichtliche Dokumentation.

		\item Machen Sie sich Gedanken über die Ausgabe ihres Programmes.
		Für die Visualisierung des Geschwindigkeitsfeldes wird ein Programm benötigt, welches zweidimensionale Vektordaten darstellen kann.

		\item Analysieren Sie die Zeitevolution der Flüssigkeit.
		Achten Sie hierbei auf die Ausbildung von Wirbeln und auf den sich einstellenden stationären Endzustand.

		\item Studieren Sie das Verhalten unterschiedlich zäher Flüssigkeiten, indem Sie die Reynoldszahl variieren.
		Wie wirkt sich dies auf die Wirbelbildung und den stationären Endzustand aus?

		\item Zur Validierung Ihres Codes vergleichen Sie Ihre Ergebnisse mit denen aus der Literatur bekannten Simulationen.

		\item Nachdem Sie sich von der Richtigkeit überzeugt haben, experimentieren Sie mit	unterschiedlichen Boxgeometrien.
		Wie verhält sich die Flüssigkeit für rechteckige Geometrien?

		\item Gehen Sie am oberen Rand von der konstanten Geschwindigkeitskomponente in eine sich zeitlich periodische Ändernde über.

	\end{itemize}

% section aufgaben (end)