\section{Mathematische und physikalische Grundlagen} % (fold)
\label{sec:grundlagen}


	\subsection{Die Navier-Stokes Gleichungen} % (fold)
	\label{sub:die_navier_stokes_gleichungen}

		Die Navier-Stokes Gleichungen lassen sich aus den Eulergleichungen wie sie aus der Thermodynamik bekannt sind ableiten.
		Für inkompressible Flüssigkeiten ($\rho = \mathrm{const}$) bilden sie folgendes System partieller Differentialgleichungen:

		\begin{alignat*}{3}
			\frac{\partial \vec{u}}{\partial t}  + (\vec{u}\cdot\nabla)\vec{u} + \nabla p &= \frac{1}{Re} \nabla \vec{u} + \vec{g} \hfill\text{ (Impulsgleichung)} \\
			\nabla \cdot \vec{u} &= 0 \hfill \text{ (Kontinuitätsgleichung)}
		\end{alignat*}

		Dieses gilt es nun mittels numerischer Verfahren zu berechnen.
		$Re$ bezeichnet hierbei die Reynoldszahl, eine dimensionslose Zahl, die zur Skalierung von Modellen verwendet wird und sich wie folgt ergibt:

		\[ Re = \frac{\rho \cdot v \cdot d}{\eta} \]

		In der Gleichung steht $v$ für die Geschwindigkeit der Strömung, $d$ für eine charakteristische Länge und $\eta$ für die dynamische Viskosität des Fluids.
		Im Programm wird $Re$ zu einer Konstanten, die Aussagen über das Strömungsverhalten liefert.
		Kleine Reynoldszahlen bedeuten dann sehr viskose Fluide und meist laminare Strömung, Fluide mit großen Reynoldszahlen zeigen meist turbulente Strömungen und müssen mit mehr Aufwand simuliert werden.
		\\Da wir im vorliegenden Fall nur zweidimensionale Probleme betrachten wollen, behandeln wir die Entwicklung der Geschwindikeit $\vec{u}$ komponentenweise in der Form:

		\[ \vec{u} = u\vec{e_x} + v\vec{e_y} \]

		Für diese skalaren Größen lauten die Zeitevolutionen dann:

		\[ \frac{\partial u}{\partial t} + \frac{\partial p}{\partial x} = \frac{1}{Re} \curvb{ \frac{\partial ^{2} u}{\partial x ^{2}} + \frac{\partial ^{2} u}{\partial y ^{2}}} 
		- \frac{ \partial (u^{2}) }{ \partial x} - \frac{ \partial (uv) }{ \partial y} + g_x\]

		\[ \frac{\partial v}{\partial t} + \frac{\partial p}{\partial y} = \frac{1}{Re} \curvb{ \frac{\partial ^{2} v}{\partial x ^{2}} + \frac{\partial ^{2} v}{\partial y ^{2}}} 
		- \frac{ \partial (uv) }{ \partial x} - \frac{ \partial (v^{2}) }{ \partial y} + g_y\]

		Die Kontinuumsgleichung lässt sich dann einfach in der Form schreiben:

		\[ \frac{\partial u}{\partial x} + \frac{\partial v}{\partial y} = 0\]

	% subsection die_navier_stokes_gleichungen (end)


	% \subsection{Danksagung} % (fold)
	% \label{sub:danksagung}

	% 	\begin{figure}[H]
	% 		\center
	% 		\includegraphics[scale = 0.35]{adrienne-bubble-blower-photodromm-09-670x1005.jpg}
	% 		\caption{Visualisierung des Beispielvektorfeldes $\vec{v}$ durch das erstellte Programm. \\ Vergleiche mit QUELLE.}
	% 		\label{fig:example view}
	% 	\end{figure}
	
	% % subsection danksagung (end)

	\subsection{Randbedingungen} % (fold)
	\label{sub:randbedingungen}
	
		Um die Randbedingungen zu formulieren, zerlegen wir die Geschwindigkeitsvektoren an den Räandern in ihre tangentialen und normalen Komponten zum Rand.
		Wir bezeichnen diese mit $\varphi _t$ und $\varphi _n$.
		Die Randbedingungen definieren wir dann mit Hilfe dieser Komponenten und deren Normalenbleitungen.
		In diesem Versuch betrachten wir nur den einfachen Fall von senkrechten und waagerechten Randstücken.

		\begin{itemize}
			\item 
			Somit ergeben sich für senkrechte Randstücke:
			\[ \varphi _n = u, \qquad \varphi_t = v, \qquad  \frac{\partial \varphi_n}{\partial n} = \frac{\partial u}{\partial x}, 
			\qquad \frac{\partial \varphi_t}{\partial n} = \frac{\partial v}{\partial x}\]

			\item
			Für waagerechte Randstücke folgt:
			\[ \varphi _n = v, \qquad \varphi_t = u, \qquad  \frac{\partial \varphi_n}{\partial n} = \frac{\partial v}{\partial y}, 
			\qquad \frac{\partial \varphi_t}{\partial n} = \frac{\partial u}{\partial y}\]

		\end{itemize}

		Mit Hilfe dieser Definitionen lassen sich dann z.B. die Haftbedingung formulieren.
		Diese gilt, wenn die Grenzen undurchlässig und star sind. 
		Flüssigkeit kann dann weder in noch aus dem Gebiet $\Omega$ fließen und ruht am Rand.
			
		\[ \varphi_n (x, y) = 0, \qquad \varphi_t (x, y) = 0 \]
		

	% subsection randbedingungen (end)

% section grundlagen (end)