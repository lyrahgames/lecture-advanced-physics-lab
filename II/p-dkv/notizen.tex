Auch der Intensitätsverlauf der Bremsstrahlungs-Schatten lässt sich auswerten.
Im Diagramm !!!!!!!!!! ist der Verlauf der Schwärzung entlang der nullten Schichtlinie für einen kleinen Ausschnitt um das Austrittsfenster gezeigt, die genau Position des untersuchten Gebiets kann man in Abbildung 333333333 sehen.


Die Schwärzung des Films erfolgt in der Regel proportional zur Aufnahmezeit und Intensität, wenn wir Überbelichtung vermeiden.
Aus der Bragg-Bedingung folgt, dass kleinere Wellenlängen weniger stark abgebeugt werden als große, was impliziert, dass man die x-Achse in Abbildung !!!!!!!!!!!! qualitativ als verzerrte Darstellung der Wellenlänge ansehen kann.
Hohe Intensitäten bedeuten stärkere Schwärzung und sind deshalb mit geringeren Werten von ImageSC belegt worden, deshalb wurde die y-Achse invertiert und ist als ebenfalls leicht verzerrte Intensität zu begreifen.
Demnach ergibt eine Art qualitatives Spektrum.
Ein Vergleich zeigt schnell eine gute Übereinstimmung mit den theoretischen Verlauf wie er unter Abschnitt \ref{ssub:charakteristische_rntgenlinien} beschrieben wurde.
Insbesondere lässt sich die $K_\alpha$-Linie gut erkennen, $K_\beta$ verschwindet wegen der wie weiter oben bereits erklärten wesentlich geringeren Intensität.