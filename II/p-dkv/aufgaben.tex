\section{Aufgaben} % (fold)
\label{sec:aufgaben}
	
	Bestimmen Sie die Gitterkonstante eines kubischen Kristalls mit Hilfe des Drehkristallverfahrens.

	\begin{itemize}
		\item Präparieren Sie einen Kristalle, indem Sie ihn entlang der Hauptnetzebenen spalten und dann durch ein Goniometer entlang der [100], [110] beziehungsweise [111] Drehachse ausrichten und fixieren.
		\item Fertigen Sie  eine Drehkristallaufnahme mit der jeweiligen Drehachse an und bestimmen Sie den Bravaisgittertyp durch Auswertung der Schichtlinienabstände.
		\item Indizieren Sie die Reflexe (insbesondere der nullten Schichtlinie). 
		\item Bestimmen Sie die Gitterkonstanten sowohl aus den Schichtlinienabständen als auch aus den Reflexpositionen in der nullten Schichtlinie.
		\item Identifizieren Sie durch Zuordnung der Gitterkonstanten die von Ihnen untersuchte Substanz.
		\item Vollführen Sie eine ausführliche Fehlerdiskussion der Gitterkonstanten. Berechnen Sie in diesem Zusammenhang die maximale Probenabsorption und diskutieren Sie auch deren Einfluss auf die Genauigkeit der Gitterkonstantenbestimmung.
		\item Fertigen Sie eine Aufnahme ohne Absorptionsfilter an und interpretieren Sie das Resultat.
		% \item Werten Sie das von Ihnen getestete Röntgenverfahren, das in der Literatur als „Drehkristall“ bekannt ist (wobei auch bei anderen Röntgenverfahren die Drehung bzw. Schwenkung der Probe als selbstverständlich gilt). 
	\end{itemize}

% section aufgaben (end)