\section{Zusammenfassung} % (fold)
\label{sec:zusammenfassung}

	Im Versuch ging es im Wesentlichen darum, die Umgangsweise mit dem Spektrographen und der CCD-Kamera zu lernen.
	Die genaue Messung des Sonnenspektrums erforderte eine gute Kalibrierung des CCD-Sensors, welche durch Bestimmung des Dunkelstroms und des Biaslevels und Vergleich mit bekannten Referenzspektren erreicht werden konnte.
	Das Sonnenspektrum konnte bis auf die typischen Abweichungen durch Zeit, Ort und Wetter nachvollzogen werden, indem charakteristische Absorptionslinien vermessen wurden.
	Auf diese Weise konnten etliche Elemente in der Atmosphäre von Sonne und Erde eindeutig nachgewiesen werden.
	Die Beobachtungen deckten sich in sehr guter Weise mit den Literaturwerten aus dem Anhang.

	Die erreichte Auflösung entsprach allerdings nicht der Theorie der Fraunhoferschen Beugung noch einer Eingrenzung durch den CCD-Sensor.
	Durch die endliche Spaltbreite der Spaltblende wird das Auflösungsvermögen des Spektrographen stark eingegrenzt.
	Die Spektrallinien also stellen eine Art Abbild des Spaltes dar.
	Je größer dabei die Spaltbreite ist, desto geringer ist das Auflösungsvermögen.
	Dieser Zusammenhang konnte eindeutig bestätigt werden.

	Der exponentielle Zusammenhang zwischen Temperatur und Dunkelstrom wurde in grober Näherung sichtbar.
	Ob die Abweichungen vom erwarteten Verlauf nun mehr durch Fehler in den relativ einfachen theoretischen Annahmen oder doch eher in systematischen Messfehlern liegen lässt sich nicht sicher sagen. 
	Fest steht, dass für eine genauere Dunkelstrommessung mehr Zeit während den Aufnahmen und vor allem längere Zeit zum Einstellen des thermischen Gleichgewichtes zwischen Thermosensor und CCD-Chip aufgewendet werden müsste.

	Die Linearität der Dispersionsrelation, wie sie in den Grundlagen angenommen wurde, konnte in den betrachteten Spektralbereichen bis auf eine Abweichung von $\pm 2\unit{$\lambda$}$ gut angenommen werden.
	Die Linearität des CCD-Sensors wiederum konnte erstaunlich gut bestätigt werden und zeigt die hervorragende Eignung derartiger Kameras zur Intensitätsbestimmung von optischen Signalen bis an den ms-Bereich.

	Insgesamt bleibt zu sagen, dass die Spektroskopie mittels optischen Gittern und CCD-Sensoren ein sehr gutes, genaues Mittel zur Analyse von Lichtsignalen in der modernen Experimentalphysik darstellt.
	Schon mit relativ einfachen und günstigen Geräten wie im Versuch verwendet ließen sich hohe Auflösungen und gute Ergebnisse erreichen. 


% section zusammenfassung (end)