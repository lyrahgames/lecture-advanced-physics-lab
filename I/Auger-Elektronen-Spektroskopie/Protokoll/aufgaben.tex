\subsection{Aufgabe} % (fold)
\label{sub:aufgabe}
	
	\begin{itemize}
		 \item
		 	Nehmen Sie Probenstrombilder Ihrer Proben mit verschiedenen Vergrößerungen auf.
		 \item
		 	Optimieren Sie Kontrast, Helligkeit und laterale Auflösung.
		 \item
		 	Kartieren Sie die Proben und die Teile des Probenhalters für eine (spätere) Zuordnung und
		 	Interpretation Ihrer Messungen.
	\end{itemize}
	
% subsection aufgabe (end)

\subsection{Aufgabe} % (fold)
\label{sub:aufgabe}

	\begin{itemize}
		\item
			Stellen Sie die Probenaufnahme mit einer Silberprobe horizontal und in einem sicheren Abstand zum Manipulator bei der Position z = 17 ein.
		\item
			Wählen Sie eine Elektronenenergie von 4keV, einen ausreichenden Probenstrom und eine Modulationsspannung von 2V sowie geeignete Parameter für die Abtastrate, die Lock-In-Zeitkonstante und die Empfindlichkeit.
		\item
			Messen Sie ein Spektrum über einen CMA-Spannungsbereich von 20 bis 500V und identifizieren Sie das Augersignal von Silber.
			Wiederholen Sie die Messung nur im Bereich des Silber-Auger-Signals (Breite des Spannungsbereiches 20 ... 50 eV).
			Optimieren Sie die Messung für ein ausreichendes Signal/Rausch-Verhältnis.
	\end{itemize}

% subsection aufgabe (end)

\subsection{Aufgabe} % (fold)
\label{sub:aufgabe}
	
	\begin{itemize} 
		\item
			Messen und bestimmen Sie die Abhängigkeiten des Signals von der Lock-In Phasenverschiebung, der Modulation der Rate und Zeitkonstante und ermitteln Sie optimale Werte für diese Parameter.
	\end{itemize}


% subsection aufgabe (end)

\subsection{Aufgabe} % (fold)
\label{sub:aufgabe}
	
	\begin{itemize}
		\item
			Bestimmen Sie den Geometriefaktor K und nutzen Sie diesen für die Energie-Kalibrierung Ihrer Spektren.
	\end{itemize}


% subsection aufgabe (end)

\subsection{Aufgabe} % (fold)
\label{sub:aufgabe}
	
	\begin{itemize}
		\item
			Nehmen Sie Auger-Spektren vom Probenhalter und einer Probe auf und interpretieren Sie die Spektren.
	\end{itemize}


% subsection aufgabe (end)

\subsection{Aufgabe} % (fold)
\label{sub:aufgabe}
	
	\begin{itemize}
		\item
			Messen Sie eine Probe vor und nach Ionenätzen.
		\item
			Interpretieren Sie das Ergebnis.
	\end{itemize}

% subsection aufgabe (end)