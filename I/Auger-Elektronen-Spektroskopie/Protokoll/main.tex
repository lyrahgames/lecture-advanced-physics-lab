% documentclass: article used for scientific journals, short reports, program documentation, etc
% options: fontsize 11, generate document for double sided printing, a4-paper
\documentclass[12pt, oneside, a4paper]{article}

% package for changing page layout
\usepackage{geometry}
\geometry{a4paper, lmargin=20mm, rmargin=20mm, tmargin=30mm, bmargin=20mm}
% set indentation
\setlength{\parindent}{0mm} 

% input encoding for special characters (e.g. ä,ü,ö,ß), only for non english text
% options: utf8 as encoding standard
\usepackage[utf8]{inputenc}
% package for changing used language
% options: german or default: english
\usepackage[german]{babel}

% package for math symbols, functions and environments from ams(american mathematical society)
\usepackage{amsmath}
% package for extended symbols from ams
\usepackage{amssymb}
% package for extended symbols from stmaryrd(st mary road)
%\usepackage{stmaryrd}
% package for managing pictures
\usepackage{graphicx}

% package for some extra fonts
\usepackage{txfonts}
% package for changing color of font and paper
% options: using names of default colors (e.g red, black)
\usepackage[usenames]{color}

% package for converting eps-files to pdf-files and then include them
\usepackage{epstopdf}
% use another program (ps2pdf) for converting
% !!! important: set shell_escape=1 in /etc/texmf/texmf.cnf (Linux) for allowing to use other programs
% !!!			 or use the command line with -shell-escape
\epstopdfDeclareGraphicsRule{.eps}{pdf}{.pdf}{
ps2pdf -dEPSCrop #1 \OutputFile
}

% package for reference to last page (output number of last page)
\usepackage{lastpage}
% package for using header and footer
% options: automate terms of right and left marks
\usepackage[automark]{scrpage2}
% set style for footer and header
\pagestyle{scrheadings}
% clear pagestyle for redefining
\clearscrheadfoot
% set header and footer: use <xx>head/foot[]{Text} (i...inner, o...outer, c...center, o...odd, e...even, l...left, r...right)
\ihead[]{Protokoll: \\ Auger-Elektronen-Spektroskopie}
\ohead[]{Clemens Anschütz \\ Markus Pawellek}
\cfoot[]{\pagemark/\pageref{LastPage}}

\setheadsepline[\textwidth]{0.5pt}
\setfootsepline[\textwidth]{0.5pt}

\usepackage[numbers,square,sort&compress]{natbib}
\bibliographystyle{abbrvnat}

\title{Protokoll: \\ Auger-Elektronen-Spektroskopie}
\author{Clemens Anschütz \\ Markus Pawellek}

\usepackage{float}

% begin the document
\begin{document}
	
	% \part*{\centering Protokoll \\ Auger-Elektronen-Spektroskopie}

	\maketitle
	\newpage

	\section{Aufgaben} % (fold)
	\label{sec:aufgaben}

		\section{Aufgaben} % (fold)
\label{sec:aufgaben}

	\begin{itemize}
		\item
			Messen Sie das Ausleserauschen des CCD-Detektors anhand von Bias-Aufnahmen.
		\item 
			Bestimmen Sie den Dunkelstrom des CCD-Detektors bei verschiedenen Detektortemperaturen mit Hilfe von Dunkelbildern mit fester Belichtungszeit. 
		\item
			Bestimmen Sie aus der Temperaturabhängigkeit die Bandlückenenergie des Halbleitermaterials der CCD.
		\item
			Überprüfen Sie die Linearität des CCD-Detektors.
		\item
			Bestimmen Sie die Eigenschaften des Gitterspektrographen, indem Sie durch die Aufnahme eines bekannten Spektrums die Dispersion und das spektrale Auflösungsvermögen für jeden der drei Spalte berechnen.
		\item
			Nehmen Sie ein Sonnenspektrum mithilfe des Gitterspektrographen auf und analysieren Sie es.
	\end{itemize}

% section aufgaben (end)

	% section aufgaben (end)

	\newpage

	\section{Grundlagen} % (fold)
	\label{sec:grundlagen}
	
		\section{Grundlagen} % (fold)
\label{sec:grundlagen}

	\subsection{Interferenz und Kohärenz} % (fold)
	\label{sub:interferenz_und_koh_renz}
		
		Bringt man zwei Teilbündel der selben Lichtquelle zur Überlagerung (wie hinter einem Youngschen Doppelspalt) kommt es zur Addition der Feldstärken. 
		Weisen beide Teilbündel die gleiche Polarisation auf, was im Normalfall ohne Bauelemente die die Polarisation beeinflussen zutrifft, so kommt es am Ort der Überlagerung zu Interferenzerscheinungen, wenn die Teilbündel eine Phasendifferenz $\varphi$ besitzen.
		Da nur gleich polarisiertes Licht interferieren kann, wird die elektrische Feldstärke $E$ im folgenden immer als skalare Größe behandelt.
		\[ E_1(t) = \hat{E}_1 \cos(\omega t),\qquad E_2(t) = \hat{E}_2 \cos(\omega t + \varphi) \]%
		Für die Intensität der interferierenden Strahlen folgt dann:
		\begin{alignat*}{3}
			\implies \quad I &=&&\ \overline{\boxb{E_1(t) + E_2(t)}^2} \\
				&=&&\  \hat{E}^2_1\ \overline{\cos^2(\omega t)} + \hat{E}^2_2\ \overline{\cos^2(\omega t)} \\
				& &&\ + 2\hat{E}_1 \hat{E}_2\ \overline{\cos(\omega t)\cos(\omega t + \varphi)}
		\end{alignat*}
		Für den zeitlich gemittelten Anteil gilt nun:
		\begin{alignat*}{3}
			\overline{\cos(\omega t)\cos(\omega t + \varphi)} &=&&\ \overline{\cos(\omega t)\boxb{\cos (\omega t) \cos \varphi - \sin (\omega t) \sin \varphi } }\\
				&=&&\ \overline{\cos^2 (\omega t) \cos \varphi} - \overline{\cos (\omega t) \sin (\omega t) \sin \varphi} \\
				&=&&\ \tfrac{1}{2}\cos \varphi
		\end{alignat*}
		 \[ \implies \quad I = I_1 + I_2 + 2 \sqrt{I_1 I_2} \cos \varphi \]
		 Für gleiche Intensitäten $I_1 = I_2 = I_0$ kann die Formel vereinfacht werden:
		 \[ I = 2 I_0 + 2 I_0 \cos\varphi = 2 I_0 \curvb{1 + \cos\varphi} \]

		 Voraussetzung für die Interferenz von Lichtstrahlen ist ihre Kohärenz.
		 Zwei Teilwellen werden kohärent genannt, wenn sie über eine konstante Phasendifferenz in Beziehung stehen.
		 Man unterscheidet räumliche und zeitliche Kohärenz.
		 Die Kohärenzzeit $t_\m{koh} $ ist die Spanne, in der sich die Phasendifferenz nicht ändert, bzw. genauer, in der sie sich um weniger als $2 \pi $ ändert.

		 Jeden Wellenzug kann man als eine Überlagerung monochromatischer Schwingungen, also als seine Fouriertransformation darstellen:

		 \[ E(\vec{r}, t) = \Integral{f_0-\Delta f/2}{f_0+\Delta f/2}{ \tilde{E}(\vec{r}, \omega) e^{-i \omega t} }{\omega} \]

		 Nach den Regeln der Fouriertransformation ist ein Wellenzug nun zeitlich umso kürzer, je breiter sein Spektrum, d.h. je größer $\Delta f$ ist.
		 Da zwei Wellenzüge generell inkohärent sind, ist also die Kohärenzlänge $t_\m{koh}$ maximal so lang wie ein Wellenzug.
		 Es gilt der einfache Zusammenhang:

		 \[ t_\m{koh} = \frac{1}{\Delta f} \]

		 Streng monochromatische Strahlung hat demnach als Fouriertransfomierte einen Delta-Peak und eine unendliche Kohärenzzeit und -länge.
		 Die Kohärenzlänge $l_\m{koh}$ ergibt sich mit der Phasengeschwindigkeit $c$ zu:

		 \[l_\m{koh} = c\ t_\m{koh} \]

	% subsection interferenz_und_koh_renz (end)


	\subsection{Prinzip des Michelson-Interferometers} % (fold)
	\label{sub:prinzip_des_michelson_interferometers}

		\urldef{\mifscheme}\url|https://upload.wikimedia.org/wikipedia/commons/thumb/b/b0/Michelson_interferometer_white_light.svg/2000px-Michelson_interferometer_white_light.svg.png|

		\begin{figure}[htb]
			\centering
			\includegraphics[scale=0.16]{images/mif-schema.png}
			\caption{Schema zum Aufbau des verwendeten Michelson-Interferometers \\ Quelle: \\ \mifscheme}
			\label{fig:grundlagen-mif-schema}
		\end{figure}

		Wie in Abbildung \ref{fig:grundlagen-mif-schema} zu sehen, durchläuft das Licht im Michelson-Interferometer (MIF) ausgehend von der Quelle zunächst den Kollimator um dann auf einen 50/50-Strahlteiler zu treffen, der zwei Bündel möglichst gleicher Intensität zu den Spiegeln Spiegel 1 und Spiegel 2 führt.
		Je genauer die beiden Intensitäten übereinstimmen, desto größer ist der Kontrast zwischen Interferenzminimum und -maximum. 
		Spiegel 2 ist verstellbar und somit kann die optische Wegdifferenz der beiden Teilstrahlen beliebig eingestellt werden.
		Nach erneuter Ablenkung durch den Strahlteiler gelangt der eine Teil der Lichtbündel zurück zur Quelle, der andere erreicht den Ausgang und wird über das Objektiv auf eine Fläche abgebildet.
		Je nach Art des Strahlteilers kann eine Kompensationsplatte in einem der beiden Strahlengänge erforderlich sein, um zu verhindern, dass nur einer der Teilstrahlen z.B. eine Strecke in Glas zurücklegt und so von vorneherein eine ungewollte Wegdifferenz aufweist.
		Beträgt die Spiegelverschiebung $\Delta x$ weniger als die halbe Kohärenzlänge, so lassen sich auf dem Schirm je nach Stellung der Spiegel unterschiedliche Interferenzerscheinungen einstellen und beobachten.
	
	% subsection prinzip_des_michelson_interferometers (end)

	\subsection{Interferenzen gleicher Neigung} % (fold)
	\label{sub:interferenzen_gleicher_neigung_haidinger_ringe}

		Bei idealen Bauteilen und idealer Justierung, d.h. plane Spiegel, keine Verzeichnung durch Linsen und Kollimator, exakt rechtwinklige Ausrichtung der Spiegel, homogenen Strahlenbündel usw. sind auf dem Schirm konzentrische Kreise zu erkennen, genannt Haidinger Ringe.
		Analog lasssen sich zwei Lichtquellen L1 und L2 im Abstand $2 \Delta x$ auf einer Achse vor dem Schirm betrachten.
		Maxima lassen sich nun an allen Stellen finden, wo die Phasendifferenz $\varphi = 2 \pi$ beträgt, also der Wegunterschied gleich $\lambda$ ist.
		Bei gleichmäßiger Erhöhung von $\Delta x$ ziehen sich die Ringe zusammen, veringert man den Wegunterschied, quellen sie auseinander.

	% subsection interferenzen_gleicher_neigung_haidinger_ringe (end)


	\subsection{Interferenzen gleicher Dicke} % (fold)
	\label{sub:interferenzen_gleicher_dicke_fizeau_streifen}
	
		Wenn beide Spiegel in etwa die gleiche Weglänge aufweisen, so kann man durch Verkippen des einen Interferenzen gleicher Dicke, sogenannte Fizeau-Streifen erzeugen.
		Die gleichen Muster treten bei zwei ebenen Wellen auf, die im Winkel $\alpha$ zusammenlaufen.
		Aus einfachen geometrischen Überlegungen folgt für den Abstand $a$ zwischen zwei Maxima auf dem Schirm:

		\[ a = \frac{\lambda}{2 \sin\frac{\alpha}{2}} \]

		Die Gleichung berücksichtigt nur ebene Wellen und keine Skalierung durch abbildende Optik.
		Die Skalierung kann nachträglich z.B. durch vermessen der Bildgröße dirket nach dem Würfel oder durch Abmessen des Linsendurchmessers erfolgen.

	% subsection interferenzen_gleicher_dicke_fizeau_streifen (end)

	\subsection{Fourierspektroskopie} % (fold)
	\label{sub:fourierspektroskopie}
		
		Wie unter \ref{sub:interferenz_und_koh_renz} beschrieben gilt für die Intensität im Falle gleicher Strahlteilung:
		\[ I = 2 I_0 + 2  I_0 \cos\varphi \]
		Drücken wir die Phasendifferenz $\varphi$ durch die zeitliche Verschiebung $\tau$ aus, erhalten wir
		\[ \varphi = \omega \tau \]
		Die Intensität ist normalerweise über einen gewissen Frequenzbereich verteilt und lässt sich somit durch
		\[ I = \integral{0}{\infty}{i(\omega)}{\omega} \]
		ausdrücken.
		Dabei stellt $i(\omega)$ die spektrale Leistungsdichte dar.
		Ziel einer jeden Spektroskopie ist es genau diese Funktion möglichst breit und hoch aufgelößt zu bestimmen.
		Aus allen drei Formeln folgt:

		\[ I(\tau) = 2 \integral{0}{\infty}{i(\omega)}{\omega} + 2 \integral{0}{\infty}{i(\omega) \cos\omega \tau }
		{\omega} \]

		Der letzte Summand stellt hierbei die Cosinus-Fourier-Transformierte dar.
		Den Zusammenhang zur allgemeinen FT kann man durch die formale Forderung $I(\omega) = I(- \omega) $ herstellen:
		\[ \integral{- \infty}{\infty}{i(\omega) e^{i \omega \tau} }{\omega} = \integral{- \infty}{\infty}{ i(\omega) \cos\omega\tau }{\omega} \]
		Das Interferogramm bildet somit die Fourier-Transformierte (FT) des Spektrums.
		Deutlich wird die am Beispiel der Na-Doppellinie.
		Zwei eng benachbarte Frequenz-Peaks im Frequenzraum sollten transformiert eine Schwebung ergeben.
		Das dies auch der Fall ist kann man in Abschnitt \ref{ssub:parameter_und_aufl_severm_gen} sehen.
		Das ursprüngliche Spektrum kann nun durch eine FFT (Fast Fourier Transformation) numerisch berechnet werden.
		Vor allem zwei Faktoren limitieren Auflösevermögen und Genauigkeit der Fourier-Spektroskopie (FS).
		Zum einen muss man sehr genau auf Bruchteile von $\lambda$ die Motorposition einstellen und bestimmen können.
		Aus diesem Grund verwenden wir im Versuch nur sehr kleine Motorgeschwindigkeiten.
		Zu dem wird die Messung an einem bekannten Referenzsignal eines Lasers kalibriert.
		Auch sieht man, das die FS vor allem im Bereich großer Wellenlängen, also im Infraroten sehr gute Ergebnisse liefert.
		Das Auflösevermögen wird jedoch vor allem durch die endliche Spiegelverschiebung begrenzt.
		Nach den Regeln der Fouriertransformation kann man für die Auflösung $A$ abschätzen:

		\[ A = \frac{\lambda}{\Delta \lambda} \approx \frac{2 s_\m{max}}{\lambda } \]
	% subsection fourierspektroskopie (end)
% section grundlagen (end)

	% section grundlagen (end)

	\newpage

	\section{Versuchsaufbau und Durchführung} % (fold)
	\label{sec:versuchsaufbau_und_durchf_hrung}

		\section{Numerik und Durchführung} % (fold)
\label{sec:numerik_und_durchf_hrung}

	\subsection{Räumliche Diskretisierung und Gitter} % (fold)
	\label{sub:r_umliche_diskretisierung_und_gitter}
	
		Die Simulation findet im rechteckigen Gebiert $\Omega:= [0,a]\times [0,b] \subset \SR^2$ statt.
		Dieses wird in $i_{max} \cdot j_{max}$ rechteckige Zellen zerlegt.
		Die Zellen des räumlichen Gitters besitzen die Maße $\delta x \cdot \delta y$  wobei gilt:

		\[ \delta x := \frac{a}{i_{max}} \qquad \text{und} \qquad \delta y := \frac{b}{j_{max}}\]

		Die skalaren Felder $u$, $v$, und $p$ ordnen wir auf diesem Gitter wie in Abbildung \ref{fig:VerschGitter} zu sehen an.

		\begin{figure}[h]
			\center
			\fbox{\includegraphics[scale=0.5]{VerschGitter.JPG}}
			\caption{Schema für die Position der Geschwindigkeitskomponenten $u$, $v$ sowie des Drucks $p$ auf dem Gitter \\ Quelle: \cite{nsfd} }
			\label{fig:VerschGitter}
		\end{figure}

		Durch die verschobenen Gitter lassen sich mögliche Oszillationen und Uneindeutigkeiten des Drucks verhindern.
		Die grauen Zellen in Abbildung \ref{fig:Randzellen} bilden die Randzellen, sie erhalten bei Initialisierung des Programms feste Werte entsprechend den jeweiligen Randwertproblemen und behalten diese während der gesamten Simulation bei.

		\begin{figure}[h]
			\center
			\fbox{\includegraphics[scale=0.5]{Randzellen.JPG}}
			\caption{Darstellung der Randzellen mit den entsprechenden Randbedingungen \\ Quelle: \cite{nsfd} }
			\label{fig:Randzellen}
		\end{figure}

	% subsection r_umliche_diskretisierung_und_gitter (end)


	\subsection{Lösen der Impulsgleichungen} % (fold)
	\label{sub:l_sen_der_impulsgleichungen}

		Die Variablen werden zum Zeitpunkt $n$ betrachtet, die nachfolgenden Geschwindigkeitskomponenten werden mit Hilfe eines einfachen Euler-Vorwärts-Schrittes berechnet:

		\[ u^{(n+1)} = F^{(n)} - \delta t \frac{\partial p^{(n+1)}}{\partial x} \]

		\[ v^{(n+1)} = G^{(n)} - \delta t \frac{\partial p^{(n+1)}}{\partial y} \]

		Dabei sind die Funktionen $F$ und $G$ wie folgt definiert:

		\[ F = u + \delta t \boxb{ \frac{1}{Re} \curvb{ \frac{\partial ^{2} u}{\partial x ^{2}} + \frac{\partial ^{2} u}{\partial y ^{2}}} 
		- \frac{ \partial (u^{2}) }{ \partial x} - \frac{ \partial (uv) }{ \partial y} + g_x } \]

		\[ G = v + \delta t \boxb{ \frac{1}{Re} \curvb{ \frac{\partial ^{2} v}{\partial x ^{2}} + \frac{\partial ^{2} v}{\partial y ^{2}}} 
		- \frac{ \partial (uv) }{ \partial x} - \frac{ \partial (v^{2}) }{ \partial y} + g_y } \]

		Das Verfahren ist wie wir sehen expliziet in den Geschwindigkeitskomponenten und impliziet im Druck.
		Die Druckberechnung wird unter Abschnitt \ref{sub:berechnung_des_drucks} genauer beschreiben.
		Der Zeitschritt $\delta t$ muss sorgfältig gewählt werden, um die Stabilität der Simulation zu gewährleisten.
		Nach Griebel, Dornseifer und Neuhöfer sollte er wie folgendermaßen gewählt werden:

		\[ \delta t := \tau \min \curvb{ \frac{Re}{2} \frac{1}{\delta x^{-2} + \delta y^{-2}} , \frac{\delta x}{\abs{u_{max}}} , 
		\frac{\delta y}{\abs{v_{max}}} } \]

		Hierbei ist $\tau$ ein Sicherheitsfaktor zwischen 0 und 1.
		Im Programm wird er meist auf 0.5 gesetzt.

	
	% subsection l_sen_der_impulsgleichungen (end)

	\subsection{Ableitungs - Stencils} % (fold)
	\label{sub:ableitungs_stencils}
	
		Die Ableitungen werden mit Hilfe der Finiten-Differenzen-Methode berechnet.
		Dabei werden die ersten Ortsableitungen beispielsweise des Drucks durch den einfachen rechtseitigen Differenzenquotienten approximiert.

		\[ \boxb{\frac{\partial p}{\partial x}}_{i,j} = \frac{p_{i+1,j} - p_{i,j}}{\delta x}, \qquad 
		\boxb{\frac{\partial p}{\partial y}}_{i,j} = \frac{p_{i,j+1} - p_{i,j}}{\delta y} \]

		Für die zweiten Ableitungen werden standardmäßig zentrierte Differenzen verwendet.

		\[ \boxb{\frac{\partial ^2 u}{\partial x ^2}}_{i,j} = \frac{u_{i+1,j} - 2 u_{i,j} + u_{i-1,j}}{(\delta x) ^2}, \qquad 
		\boxb{\frac{\partial ^2 u}{\partial y ^2}}_{i,j} = \frac{u_{i,j+1} - 2 u_{i,j} + u_{i,j-1}}{(\delta y) ^2} \]

		Analoge Vorschriften liegen für die Ableitungen von $v$ vor.
		Die Ableitungen der gemischten und nichtlinearen Terme werden durch eine Linearkombination von zentrierten Differenzen und Donor-Cell-Stencils ermittelt.

		\begin{alignat*}{3}
			\boxb{ \frac{\partial (u ^2)}{\partial x} }_{i,j} &= \frac{1}{4 \delta x}
			\curvb{ (u_{i,j} + u_{i+1,j})^2 - (u_{i-1,j} + u_{i,j})^2 } \\
			&+ \gamma \frac{1}{4 \delta x} 
			\curvb{ \abs{u_{i,j} + u_{i+1,j}} (u_{i,j} - u_{i+1,j}) - \abs{u_{i-1,j} + u_{i,j}} (u_{i-1,j} - u_{i,j}) } \\
			\boxb{ \frac{\partial (v ^2)}{\partial y} }_{i,j} &= \frac{1}{4 \delta y}
			\curvb{ (v_{i,j} + v_{i,j+1})^2 - (v_{i,j-1} + v_{i,j})^2 } \\
			&+ \gamma \frac{1}{4 \delta y} 
			\curvb{ \abs{v_{i,j} + v_{i,j+1}} (v_{i,j} - u_{i,j+1}) - \abs{v_{i,j-1} + v_{i,j}} (v_{i,j-1} - v_{i,j}) } \\
			\boxb{ \frac{\partial (uv)}{\partial x} }_{i,j} &= \frac{1}{4 \delta x}
			\curvb{ (u_{i,j} + u_{i,j+1})(v_{i,j} + v_{i+1,j}) - (u_{i-1,j} + u_{i-1,j+1})(v_{i-1,j} + v_{i,j}) } \\
			&+ \gamma \frac{1}{4 \delta x} 
			\curvb{ \abs{u_{i,j} + u_{i,j+1}} (v_{i,j} - v_{i+1,j}) - \abs{u_{i-1,j} + u_{i-1,j+1}} (v_{i-1,j} - v_{i,j}) } \\
			\boxb{ \frac{\partial (uv)}{\partial y} }_{i,j} &= \frac{1}{4 \delta y}
			\curvb{ (u_{i,j} + u_{i,j+1})(v_{i,j} + v_{i+1,j}) - (u_{i,j-1} + u_{i,j})(v_{i,j-1} + v_{i+1,j-1}) } \\
			&+ \gamma \frac{1}{4 \delta y} 
			\curvb{ \abs{v_{i,j} + v_{i+1,j}} (u_{i,j} - u_{i,j+1}) - \abs{v_{i,j-1} + v_{i+1,j-1}} (u_{i,j-1} - u_{i,j}) } \\
		\end{alignat*}

		Der Vorfaktor $\gamma$ gewichtet hierbei die Anteile von zentrierten Differenzen und Donor-Cell-Stencils.
		Er sollte nach Hirt so gewählt werden, dass er folgende Bedingung erfüllt:

		\[ \gamma \geq \max_{ij} \curvb{ \frac{\abs{u_{i,j}} \delta t}{\delta x} , \frac{\abs{v_{i,j}} \delta t}{\delta y} } \]

		Jedoch wird er in unserem Programm zu Beginn auf einen konstanten Wert gesetzt, wie er auch in anderen Simulationen bereits verwendet wurde.

	% subsection ableitungs_stencils (end)


	\subsection{Berechnung des Drucks} % (fold)
	\label{sub:berechnung_des_drucks}

		Zur Bestimmung des Drucks verwenden wir die Kontinuitätsgleichung.

		\[ \frac{\partial u^{n+1}}{\partial x} + \frac{\partial v^{n+1}}{\partial y} = 0\]

		Zusammen mit der unter \ref{sub:l_sen_der_impulsgleichungen} angegebenen Gleichung für die Ermittlung der Geschwindigkeiten lässt sich folgende Poisson-Gleichung für den Druck ableiten:

		\[ \frac{\partial ^{2} p^{(n+1)}}{\partial x^{2}} + \frac{\partial ^{2} p^{(n+1)}}{\partial y^{2}} = 
		\frac{1}{\delta t} \curvb{ \frac{\partial F^{(n)}}{\partial x} + \frac{\partial G^{(n)}}{\partial y} } \]

		Die zweiten und ersten Ortsableituungen werden analog zu den unter \ref{sub:ableitungs_stencils} beschriebenen Ableitungen gebildet.
		Zum lösen der diskreten Poissongleichung wird hier das Successive-Over-Relaxation Verfahren (SOR) angewendet.
		Bei diesem iterativen Verfahren setzen wir den Startwert zu $p_{i,j} ^{(n)}$, jeder weitere Iterationsschritt ergibt sich dann zu:

		\[ p_{i,j} ^{it+1} = (1- \omega) p_{i,j}^{it} + \frac{\omega}{2 (\delta x)^{-2} + 2 (\delta y)^{-2}} 
		\curvb{ \frac{ p_{i+1,j}^{it} + p_{i-1,j}^{it+1} }{(\delta x)^2} + \frac{ p_{i,j+1}^{it} + p_{i,j-1}^{it+1} }{(\delta y)^2} - \text{RHS}_{i,j} } \]

		Wobei RHS die rechte Seite der Poissongleichung bezeichnet.
		In jeder Iteration wird das Residuum $r$ berechnet 

		\[ r_{i,j}^{it+1} = \frac{ p_{i+1,j}^{it+1} - 2 p_{i,j}^{it+1} + p_{i-1,j}^{it+1} }{ (\delta x)^2 } + 
		\frac{ p_{i,j+1}^{it+1} - 2 p_{i,j}^{it+1} + p_{i,j-1}^{it+1} }{ (\delta y)^2 } - \text{RHS}_{i,j}\]

		bis die Norm des Residuuums eine vorgegebene relative Toleranzgrenze unterschreitet.

		\[ \norm{r^{it+1}} < \epsilon \norm{p^{it}} \]

		Wobei die Norm im Versuch durch die Maximumsnorm gebildet wird und $\epsilon$ eine festgelegte kleine Konstante darstellt.
		Die Randzellen werden bei jedem Iterationsschritt einfach durch Kopieren der jeweiligen Nachbarzellen gefüllt.


	% subsection berechnung_des_drucks (end)

% section numerik_und_durchf_hrung (end) 
	
	% section versuchsaufbau_und_durchf_hrung (end)

	\newpage

	\section{Messergebnisse und Diskussion} % (fold)
	\label{sec:messergebnisse}

		\section{Simulation und Ergebnisse} % (fold)
\label{sec:simulation_und_ergebnisse}

	
	\subsection{Ausgabe und Visualisierung} % (fold)
	\label{sub:ausgabe_und_visualisierung}
	
		Für die Visualisierung wurde mithilfe der Bibliothek \textit{Qt v5.1.1} eine Klasse erstellt, welche durch die Übergabe eines diskretisierten Vektorfeldes, dieses automatisch darstellt.
		Dabei wird eine bestimmte Anzahl an Position zufällig gewählt.
		An diesen Position werden dann Teilausschnitte der Strömungslinien des Vektorfeldes angezeigt.
		Die Länge stellt ein natürliches Maß für die Stärke des Vektorfeldes an den jeweiligen Punkten dar.
		In Abbildung \ref{fig:example view} ist als Beispiel das folgende Vektorfeld $\vec{v}$ auf dem Bereich $[0,1]^2$ dargestellt worden.
		\[ \vec{v}(x,y) := x\vec{e_x} + \sqrt{x}\sin(3\pi x)\sin(\pi y)\vec{e_y} \]

		\begin{figure}[!htb]
			\center
			\includegraphics[scale = 0.35]{screenshots/example-01.png}
			\caption{Visualisierung des Beispielvektorfeldes $\vec{v}$ durch das erstellte Programm. \\ Vergleiche mit \cite{nsfd}.}
			\label{fig:example view}
		\end{figure}

		Für diskretisierte skalare Felder, wie den Druck, ist die selbe Klasse zuständig.
		Dabei werden verschiedene Farbwerte verwendet um die Stärke des Feldes zu verdeutlichen.
		Im Programm selbst wurden stärkere Werte durch rötlichere Farben und schwächere durch bläuliche Farben dargestellt.
		Als Beispiel sei die Visualisierung des skalaren Feldes $v_y$ durch das Programm in Abbildung \ref{fig:example view scalar} gegeben.

		\begin{figure}[!htb]
			\center
			\includegraphics[scale = 0.35]{screenshots/example-02.png}
			\caption{Visualisierung des skalaren Feldes $v_y$ durch das Programm. \\ Vergleiche mit \cite{nsfd}.}
			\label{fig:example view scalar}
		\end{figure}

	% subsection ausgabe_und_visualisierung (end)

	\subsection{Zeitevoluion einer Flüssigkeit} % (fold)
	\label{sub:zeitevoluion_einer_fl_ssigkeit}
	
		Im Folgenden wurden für das Programm die unten stehenden Parameter verwendet.
		\[ Re = 100 \quad u = 1.28 \quad i_{max} = 64 \quad j_{max} = 64 \]
		\[ a=b=1.0 \quad \tau = 0.5 \quad \gamma = 0.5 \quad \epsilon = 0.001 \quad \omega = 0.5 \]
		Diese Parameter werden beibehalten, sollte keine weitere Angabe zu Ihnen erfolgen.

		Die Nischenströmung konnte mithilfe des Programmes erfolgreich simuliert werden.
		Es wurden für vier verschiedene Zeiten die Evolution der Flüssigkeit in der Zelle aufgenommen.
		Diese sind in Abbildung zu sehen.

		\begin{figure}[!htb]
			\begin{subfigure}[b]{.5\textwidth}
				\center
				\includegraphics[scale=0.28]{screenshots/time-0030.png}
				\caption{$t=0.03$}
				\label{fig:time 01}
			\end{subfigure}
			\begin{subfigure}[b]{.5\textwidth}
				\center
				\includegraphics[scale = 0.28]{screenshots/time-0060.png}
				\caption{$t=0.06$}
				\label{fig:time 02}
			\end{subfigure}
			\begin{subfigure}[b]{.5\textwidth}
				\center
				\includegraphics[scale = 0.28]{screenshots/time-0200.png}
				\caption{$t=0.2$}
				\label{fig:time 03}
			\end{subfigure}
			\begin{subfigure}[b]{.5\textwidth}
				\center
				\includegraphics[scale = 0.28]{screenshots/time-3800.png}
				\caption{$t=3.8$ (stationärer Endzustand)}
				\label{fig:time 04}
			\end{subfigure}
			\caption{simulierte Zeitevolution des Geschwindigkeits- und Druckfeldes für $Re = 100$}
			\label{fig:time evolution}
		\end{figure}

		Diese Ergebnisse entsprechen den Ergebnissen aus \cite{nsfd} und decken sich mit der Erfahrung.
		Zu Beginn sind deutliche Druckunterschiede zu beobachten.
		Im oberen rechten Bereich entsteht durch die Aufstauung der Flüssigkeit an der rechten Wand eine Druckerhöhung.
		Analog dazu lässt sich der Druck im linken Bereich durch einen Sog erklären.
		Im Laufe der Zeit gleichen sich die Druckunterschiede aus, da die Flüssigkeit inkompressibel ist.
		Im stationären Endzustand, der nach circa $3.5\unit{s}$ erreicht war, ist der Druck im gesamten Bereich minimal (hellblaue Farbe).

		Das Geschwindigkeitsfeld bildet bereits kurz nach dem Start der Simulation einen Wirbel aus.
		Dieser ist zuerst, wie in Abbidung zu erkennen, sehr flach im oberen Bereich zu sehen.
		Mit der Zeit wird die gesamte träge Flüssigkeit aufgrund innerer Reibung beschleunigt.
		Der Wirbel wird größer bis er im Endzustand fast die gesamte Zelle einnimmt.
		Dabei verschiebt sich das Zentrum des Wirbels leicht nach rechts.
		Aufgrund der nicht rotationssymmetrischen Boxgeometrie kommt es zu Deformierungen des Wirbels an den Ecken der Zelle.
		In den unteren beiden Ecken der Zellen sind die Beträge des Geschwindigkeitsfeldes zu gering, um eine nähere Untersuchung zuzulassen.
		Es lässt sich jedoch erkennen, dass der Wirbel dort abbricht.

		In der rechten oberen Ecke verlassen einige Pfeile die Box.
		Dies würde den Haftbedingungen widersprechen.
		Jedoch sind diese Fehler auf numerische Rundungsfehler bei der Darstellung und endliche Ortsauflösung zurückzuführen, da die Gesamtlösung dadurch nicht verändert wird.

	% subsection zeitevoluion_einer_fl_ssigkeit (end)


	\subsection{Einfluss der Reynoldszahl} % (fold)
	\label{sub:einfluss_der_reynoldszahl}

		Die Simulationen wurden nun mit steigender Reynoldszahl durchgeführt.
		Alle weiteren Parameter wurden konstant gelassen.
		Im Folgenden sollen nur noch die stationären Endzustände der Geschwindigkeitsfelder betrachtet werden.

		Während für Reynoldszahlen von $1$ bis $10$ die Ergebnisse qualitativ gleich sind, erkennt eine deutliche Verschiebung und Deformierung des Wirbels ab $Re = 100$.
		Kleinere Reynoldszahlen bedeuten eine höhere Viskosität.
		In den ersten beiden Fällen ist also die innere Reibung der Flüssigkeiten so groß, dass asymmetrische Turbulenzen verhindert werden.

		\begin{figure}[!htb]
			\begin{subfigure}[b]{.5\textwidth}
				\center
				\includegraphics[scale = 0.28]{screenshots/re-1.png}
				\caption{$Re=1$}
				\label{fig:re 1}
			\end{subfigure}
			\begin{subfigure}[b]{.5\textwidth}
				\center
				\includegraphics[scale = 0.28]{screenshots/re-10.png}
				\caption{$Re=10$}
				\label{fig:re 10}
			\end{subfigure}
			
			\begin{subfigure}[b]{.5\textwidth}
				\center
				\includegraphics[scale = 0.28]{screenshots/time-3800.png}
				\caption{$Re=100$}
				\label{fig:re 100}
			\end{subfigure}
			\begin{subfigure}[b]{.5\textwidth}
				\center
				\includegraphics[scale = 0.28]{screenshots/re-1000-64.png}
				\caption{$Re=1000$}
				\label{fig:re 1000 64}
			\end{subfigure}
			\caption{simulierte Strömungslinien im stationären Endzustand für verschiedene Reynoldszahlen}
			\label{fig:steady state re}
		\end{figure}

		Im Falle von $Re = 1000$ ist deutlich erkennbar, dass ein Fehler in der Berechnung aufgetreten ist.
		Hier stößt man mit der Ortsauflösung von $64\times64$ Gitterpunkten und der daraus folgenden Zeitauflösung an die Grenzen der Simulation.
		In dem Programm wird eine $32$-bit Auflösung der reellen Zahlen verwendet.
		Diese besitzt eine relative Genauigkeit von rund $10^{-6}$.
		Mit einer $64$-bit Auflösung könnte in diesem Falle nun ein besseres Ergebnis erzielt werden.
		Auch eine zu groß gewählte Randgeschwindigkeit kann zu einer numerischen Instabilität führen.
		Für $Re=1000$ haben wir nun die Anzahl der Gitterpunkte auf $512\times 512$ erhöht und die Geschwindigkeit auf $1.0$ gesetzt.
		Es war somit möglich eine stabile numerische Berechnung zu realisieren.
		Dies ist jedoch mit einem erheblich größeren Rechen- und Zeitaufwand verbunden.
		Die genauen Ergebnisse der Simulation sind in den Abbildungen \ref{fig:time re 1000 1} und \ref{fig:time re 1000 2} zu sehen.
		Diese Zeitevolution stimmt mit den Ergebnissen aus \cite{nsfd} ohne Einschränkungen überein (siehe Anhang).

		\begin{figure}[!htb]
			\centering
			\begin{subfigure}[b]{.5\textwidth}
				\includegraphics[scale = 0.28]{screenshots/re-1000-512-00520.png}
				\caption{$t=0.520$}
			\end{subfigure}%
			\begin{subfigure}[b]{.5\textwidth}
				\includegraphics[scale = 0.28]{screenshots/re-1000-512-01069.png}
				\caption{$t=1.069$}
			\end{subfigure}

			\begin{subfigure}[b]{.5\textwidth}
				\includegraphics[scale = 0.28]{screenshots/re-1000-512-01555.png}
				\caption{$t=1.555$}
			\end{subfigure}%
			\begin{subfigure}[b]{.5\textwidth}
				\includegraphics[scale = 0.28]{screenshots/re-1000-512-02170.png}
				\caption{$t=2.170$}
			\end{subfigure}

			\begin{subfigure}[b]{.5\textwidth}
				\includegraphics[scale = 0.28]{screenshots/re-1000-512-02405.png}
				\caption{$t=2.405$}
			\end{subfigure}%
			\begin{subfigure}[b]{.5\textwidth}
				\includegraphics[scale = 0.28]{screenshots/re-1000-512-04758.png}
				\caption{$t=4.758$}
			\end{subfigure}
			\caption{simulierte Zeitevolution der Strömungslinien und des Druckfeldes für $Re=1000$ auf einem $512\times 512$-Gitter für verschiedene Zeitparameter $t$}
			\label{fig:time re 1000 1}
		\end{figure}

		\begin{figure}[!htb]
			\centering
			\begin{subfigure}[b]{.5\textwidth}
				\includegraphics[scale = 0.28]{screenshots/re-1000-512-06482.png}
				\caption{$t=6.482$}
			\end{subfigure}%
			\begin{subfigure}[b]{.5\textwidth}
				\includegraphics[scale = 0.28]{screenshots/re-1000-512-07612.png}
				\caption{$t=7.612$}
			\end{subfigure}

			\begin{subfigure}[b]{.5\textwidth}
				\includegraphics[scale = 0.28]{screenshots/re-1000-512-08419.png}
				\caption{$t=8.419$}
			\end{subfigure}%
			\begin{subfigure}[b]{.5\textwidth}
				\includegraphics[scale = 0.28]{screenshots/re-1000-512-10164.png}
				\caption{$t=10.164$}
			\end{subfigure}

			\begin{subfigure}[b]{.5\textwidth}
				\includegraphics[scale = 0.28]{screenshots/re-1000-512-11408.png}
				\caption{$t=11.408$}
			\end{subfigure}%
			\begin{subfigure}[b]{.5\textwidth}
				\includegraphics[scale = 0.28]{screenshots/re-1000-512-12475.png}
				\caption{$t=12.475$}
			\end{subfigure}
			\caption{simulierte Zeitevolution der Strömungslinien und des Druckfeldes für $Re=1000$ auf einem $512\times 512$-Gitter für verschiedene Zeitparameter $t$}
			\label{fig:time re 1000 2}
		\end{figure}

	% subsection einfluss_der_reynoldszahl (end)


	\subsection{Untersuchung von rechtwinkligen Geometrien} % (fold)
	\label{sub:untersuchung_von_rechtwinkligen_geometrien}
	
		Im Folgenden werden wieder nur stationäre Zustände des Strömungsfeldes betrachtet.
		Dabei sei wieder $Re=100$ und die Größe des Gitters $64\times 64$.

		Die Simulationen für rechteckige Boxgeometrien sind in den Abbildungen \ref{fig:box 1} und \ref{fig:box 2} zu sehen.
		Es wird deutlich, dass sich durch die Wahl von $a\geq b$ die Lösung qualitativ kaum von der ursprünglichen Lösung unterscheidet.
		Der entstehende Wirbel scheint einfach mit dem Parameter $a$ skaliert zu werden.
		Ein wesentlicher Unterschied besteht hier in der Deformation des Wirbels, welche aber eine logische Konsequenz aus der Änderung der Boxgeometrie ist.

		\begin{figure}[!hptb]
			\centering
			\begin{subfigure}[b]{.5\textwidth}
				\center
				\includegraphics[scale = 0.30]{screenshots/box-05-10.png}
				\caption{$a=0.5, b=1.0$}
			\end{subfigure}%
			\begin{subfigure}[b]{.5\textwidth}
				\center
				\includegraphics[scale = 0.30]{screenshots/box-05-15.png}
				\caption{$a=0.5, b=1.5$}
			\end{subfigure}
			\caption{Simulation des stationären Strömungsfeldes für verschiedene Boxgeometrien}
			\label{fig:box 1}
		\end{figure}

		Für den Fall $a<b$ ist aber nun in Abbildung \ref{fig:box 1} eine etwas andere Strömung zu sehen.
		Der gesamte Wirbel wird weder deformiert noch skaliert.
		Er behält in etwa seine Größe bei.

		\begin{figure}[!hptb]
			% \centering
			\begin{subfigure}[b]{.5\textwidth}
				\center
				\includegraphics[scale = 0.32]{screenshots/box-10-05.png}
				\caption{$a=1.0, b=0.5$}
			\end{subfigure}

			\begin{subfigure}[b]{.5\textwidth}
				\center
				\includegraphics[scale = 0.32]{screenshots/box-15-05.png}
				\caption{$a=1.5, b=0.5$}
			\end{subfigure}
			\caption{Simulation des stationären Strömungsfeldes für verschiedene Boxgeometrien}
			\label{fig:box 2}
		\end{figure}

	% subsection untersuchung_von_rechtwinkligen_geometrien (end)


	\subsection{Periodische Anregung} % (fold)
	\label{sub:periodische_anregung}

		Für diesen Versuchsteil wurde für die Randgeschwindigkeit die folgende periodische Funktion verwendet.
		\[ \cos 3t \]
		Dabei beschreibt $t$ wieder den Zeitparameter der Simulation.
		Die Ergebnisse sind in den Abbildungen \ref{fig:periodic 1} und \ref{fig:periodic 2} zu sehen.
	
		\begin{figure}[!htb]
			\centering
			\begin{subfigure}[b]{.5\textwidth}
				\includegraphics[scale = 0.28]{screenshots/periodic-00063.png}
				\caption{$t=0.063$}
			\end{subfigure}%
			\begin{subfigure}[b]{.5\textwidth}
				\includegraphics[scale = 0.28]{screenshots/periodic-00286.png}
				\caption{$t=0.286$}
			\end{subfigure}

			\begin{subfigure}[b]{.5\textwidth}
				\includegraphics[scale = 0.28]{screenshots/periodic-00507.png}
				\caption{$t=0.507$}
			\end{subfigure}%
			\begin{subfigure}[b]{.5\textwidth}
				\includegraphics[scale = 0.28]{screenshots/periodic-00702.png}
				\caption{$t=0.702$}
			\end{subfigure}

			\begin{subfigure}[b]{.5\textwidth}
				\includegraphics[scale = 0.28]{screenshots/periodic-00944.png}
				\caption{$t=0.944$}
			\end{subfigure}%
			\begin{subfigure}[b]{.5\textwidth}
				\includegraphics[scale = 0.28]{screenshots/periodic-01162.png}
				\caption{$t=1.162$}
			\end{subfigure}
			\caption{simulierte Zeitevolution Strömungslinien und des Druckfeldes, welche periodisch angeregt werden, für verschiedene Zeitparameter $t$}
			\label{fig:periodic 1}
		\end{figure}

		\begin{figure}[!htb]
			\centering
			\begin{subfigure}[b]{.5\textwidth}
				\includegraphics[scale = 0.28]{screenshots/periodic-01442.png}
				\caption{$t=1.442$}
			\end{subfigure}%
			\begin{subfigure}[b]{.5\textwidth}
				\includegraphics[scale = 0.28]{screenshots/periodic-01606.png}
				\caption{$t=1.606$}
			\end{subfigure}

			\begin{subfigure}[b]{.5\textwidth}
				\includegraphics[scale = 0.28]{screenshots/periodic-01735.png}
				\caption{$t=1.735$}
			\end{subfigure}%
			\begin{subfigure}[b]{.5\textwidth}
				\includegraphics[scale = 0.28]{screenshots/periodic-01921.png}
				\caption{$t=1.921$}
			\end{subfigure}

			\begin{subfigure}[b]{.5\textwidth}
				\includegraphics[scale = 0.28]{screenshots/periodic-02299.png}
				\caption{$t=2.299$}
			\end{subfigure}%
			\begin{subfigure}[b]{.5\textwidth}
				\includegraphics[scale = 0.28]{screenshots/periodic-02573.png}
				\caption{$t=2.573$}
			\end{subfigure}
			\caption{simulierte Zeitevolution Strömungslinien und des Druckfeldes, welche periodisch angeregt werden, für verschiedene Zeitparameter $t$}
			\label{fig:periodic 2}
		\end{figure}

		Wie zu erwarten war wechseln die Drehrichtungen des Wirbels, je nachdem in welche Richtung die Randgeschwindigkeit zeigt.
		Es ist wichtig zu bemerken, dass beim Wechsel der Geschwindigkeitsrichtung ein enormer Druckanstieg zu beobachten ist.

	% subsection periodische_anregung (end)

% section simulation_und_ergebnisse (end)
	
	% section messergebnisse (end)

	\newpage

	\section{Zusammenfassung} % (fold)
	\label{sec:zusammenfassung}
	
		\section{Zusammenfassung} % (fold)
\label{sec:zusammenfassung}

	Im Versuch ging es im Wesentlichen darum, die Umgangsweise mit dem Spektrographen und der CCD-Kamera zu lernen.
	Die genaue Messung des Sonnenspektrums erforderte eine gute Kalibrierung des CCD-Sensors, welche durch Bestimmung des Dunkelstroms und des Biaslevels und Vergleich mit bekannten Referenzspektren erreicht werden konnte.
	Das Sonnenspektrum konnte bis auf die typischen Abweichungen durch Zeit, Ort und Wetter nachvollzogen werden, indem charakteristische Absorptionslinien vermessen wurden.
	Auf diese Weise konnten etliche Elemente in der Atmosphäre von Sonne und Erde eindeutig nachgewiesen werden.
	Die Beobachtungen deckten sich in sehr guter Weise mit den Literaturwerten aus dem Anhang.

	Die erreichte Auflösung entsprach allerdings nicht der Theorie der Fraunhoferschen Beugung noch einer Eingrenzung durch den CCD-Sensor.
	Durch die endliche Spaltbreite der Spaltblende wird das Auflösungsvermögen des Spektrographen stark eingegrenzt.
	Die Spektrallinien also stellen eine Art Abbild des Spaltes dar.
	Je größer dabei die Spaltbreite ist, desto geringer ist das Auflösungsvermögen.
	Dieser Zusammenhang konnte eindeutig bestätigt werden.

	Der exponentielle Zusammenhang zwischen Temperatur und Dunkelstrom wurde in grober Näherung sichtbar.
	Ob die Abweichungen vom erwarteten Verlauf nun mehr durch Fehler in den relativ einfachen theoretischen Annahmen oder doch eher in systematischen Messfehlern liegen lässt sich nicht sicher sagen. 
	Fest steht, dass für eine genauere Dunkelstrommessung mehr Zeit während den Aufnahmen und vor allem längere Zeit zum Einstellen des thermischen Gleichgewichtes zwischen Thermosensor und CCD-Chip aufgewendet werden müsste.

	Die Linearität der Dispersionsrelation, wie sie in den Grundlagen angenommen wurde, konnte in den betrachteten Spektralbereichen bis auf eine Abweichung von $\pm 2\unit{$\lambda$}$ gut angenommen werden.
	Die Linearität des CCD-Sensors wiederum konnte erstaunlich gut bestätigt werden und zeigt die hervorragende Eignung derartiger Kameras zur Intensitätsbestimmung von optischen Signalen bis an den ms-Bereich.

	Insgesamt bleibt zu sagen, dass die Spektroskopie mittels optischen Gittern und CCD-Sensoren ein sehr gutes, genaues Mittel zur Analyse von Lichtsignalen in der modernen Experimentalphysik darstellt.
	Schon mit relativ einfachen und günstigen Geräten wie im Versuch verwendet ließen sich hohe Auflösungen und gute Ergebnisse erreichen. 


% section zusammenfassung (end)

	% section zusammenfassung (end)

	\newpage

	\bibliography{ref}

	\newpage

	\section*{Anhang} % (fold)
	\label{sec:anhang}
	
		\subsection*{Vergleichsspektren aus \cite{handbook}} % (fold)
		\label{sub:vergleichsspektren_aus_cite_handbook}
		
			\begin{figure}[H]
				\center
				\includegraphics[scale=0.15]{al-spektrum.jpg}
				\caption{Aluminium-Spektrum}
			\end{figure}

			\begin{figure}[H]
				\center
				\includegraphics[scale=0.15]{o-spektrum.jpg}
				\caption{Sauerstoff-Spektrum}
			\end{figure}

			\begin{figure}[H]
				\center
				\includegraphics[scale=0.15]{c-spektrum.jpg}
				\caption{Kohlenstoff-Spektrum}
			\end{figure}

			\begin{figure}[H]
				\center
				\includegraphics[scale=0.15]{si-spektrum.jpg}
				\caption{Silizium-Spektrum}
			\end{figure}

			\begin{figure}[H]
				\center
				\includegraphics[scale=0.15]{ag-spektrum.jpg}
				\caption{Silber-Spektrum}
			\end{figure}

		% subsection vergleichsspektren_aus_cite_handbook (end)

	% section anhang (end)
	
\end{document}

