\input{pre}

\DeclareMathOperator{\sign}{sign}

\AtBeginSection[]{
  \begin{frame}
  \vfill
  \centering
  \begin{beamercolorbox}[sep=8pt,center,shadow=true,rounded=true]{title}
  	\hrule
  	\bigskip
    \usebeamerfont{title}\insertsectionhead\par%
    \bigskip
    \hrule
  \end{beamercolorbox}
  \vfill
  \end{frame}
}

\begin{document}

	\begin{frame}
		\begin{figure}
			\centering
			\includegraphics[scale=0.25]{images/Porsche-919-Hybrid-wind-tunnel-testing.jpg}
			\caption{Luftströmungen eines fahrenden Rennwagens \\ \scriptsize Quelle:\\ \url{http://autonetmagz.net/wp-content/uploads/2014/03/Porsche-919-Hybrid-wind-tunnel-testing.jpg}}
		\end{figure}
	\end{frame}
	
	\frame{\hrule\maketitle\hrule}
	\frame{\frametitle{Gliederung} \begin{adjustwidth}{1em}{}\tableofcontents\end{adjustwidth}}


	\section{Grundlagen} % (fold)
	\label{sec:grundlagen}

		\subsection{Erinnerung} % (fold)
		\label{sub:erinnerung}
		
			\begin{frame}
				\frametitle{Erinnerung: Thermodynamik}
				\begin{enumerate}[label=(\roman*)]
					\item Kontinuitätsgleichung:
						\[ \tcboxmath{\partial_t\varrho + \nabla\cdot\curvb{ \varrho\vec{v} } = 0} \]
					\item Impulsgleichung:
						\[ \tcboxmath{\varrho \boxb{ \partial_t \vec{v} + \curvb{ \vec{v}\cdot\nabla }\vec{v} } = \vec{f} + \nabla\cdot\sigma} \]
				\end{enumerate}
			\end{frame}
		
		% subsection erinnerung (end)


		\subsection{Navier-Stokes-Gleichungen} % (fold)
		\label{sub:navier_stokes_gleichungen}
		
			\begin{frame}
				\textbf{Annahme:}\\
				Betrachtung eines einzelnen newtonschen Fluids (z.B. Wasser, Öl, Luft, etc.).
				\vfill
				\pause
				\begin{tcolorbox}[title = Navier-Stokes Gleichungen]
					\begin{alignat*}{3}
						\partial_t\varrho + \nabla\cdot\curvb{ \varrho\vec{v} } &=&& \ 0 \\
						\varrho \boxb{ \partial_t \vec{v} + \curvb{ \vec{v}\cdot\nabla }\vec{v} } &=&& \ \vec{f} - \nabla p + \eta\Delta\vec{v} + \curvb{ \tfrac{\eta}{3}+\xi }\nabla\curvb{ \nabla\cdot\vec{v} }
					\end{alignat*}
				\end{tcolorbox}
			\end{frame}

			\begin{frame}[label=nsg]
				\begin{tcolorbox}[title = dimensionslose Navier-Stokes Gleichungen inkompressibler Flüssigkeiten]
					\begin{alignat*}{3}
						\nabla\cdot\vec{v} &=&& \ 0 \\
						\partial_t \vec{v} + \curvb{ \vec{v}\cdot\nabla }\vec{v} &=&& \ \vec{g} - \nabla p + \tfrac{1}{Re}\Delta\vec{v}
					\end{alignat*}
				\end{tcolorbox}
			\end{frame}
		
		% subsection navier_stokes_gleichungen (end)

	% section grundlagen (end)

	\section{Numerische Verfahren} % (fold)
	\label{sec:numerische_verfahren}
	
		\subsection{Diskretisierung} % (fold)
		\label{sub:diskretisierung}
		
			\begin{frame}
				\frametitle{Finite-Differenzen-Methode}
				\begin{enumerate}[label=(\roman*)]
					\item Vorwärts-Differenzenquotient
						\[ \tcboxmath{ u^\prime = \frac{u(x_{i+1}) - u(x_i)}{\delta x} + \mathcal{O}(\delta x) } \]
					\item Rückwärts-Differenzenquotient
						\[ \tcboxmath{ u^\prime = \frac{u(x_{i}) - u(x_{i-1})}{\delta x} + \mathcal{O}(\delta x) } \]
					\item zentraler Differenzenquotient
						\[ \tcboxmath{ u^\prime = \frac{u(x_{i+1}) - u(x_{i-1})}{2\delta x} + \mathcal{O}(\delta x^2) } \]
				\end{enumerate}
			\end{frame}

			\begin{frame}
				\frametitle{Verschobenes Gitter (staggered grid)}
				\begin{figure}
					\centering
					\includegraphics[scale = 0.30]{staggered-grid.png}
				\end{figure}
			\end{frame}

		% subsection diskretisierung (end)

		\subsection{Algorithmus} % (fold)
		\label{sub:algorithmus}
		
			\againframe{nsg}

		% subsection algorithmus (end)

	% section numerische_verfahren (end)

	\section{Ergebnisse} % (fold)
	\label{sec:ergebnisse}
	
		\begin{frame}
			\begin{figure}
				\centering
				\includegraphics[scale=0.4]{images/re-1000-512-00520.png}
			\end{figure}
		\end{frame}

		\begin{frame}
			\begin{figure}
				\centering
				\includegraphics[scale=0.4]{images/re-1000-512-01069.png}
			\end{figure}
		\end{frame}

		\begin{frame}
			\begin{figure}
				\centering
				\includegraphics[scale=0.4]{images/re-1000-512-01555.png}
			\end{figure}
		\end{frame}

		\begin{frame}
			\begin{figure}
				\centering
				\includegraphics[scale=0.4]{images/re-1000-512-02170.png}
			\end{figure}
		\end{frame}

		\begin{frame}
			\begin{figure}
				\centering
				\includegraphics[scale=0.4]{images/re-1000-512-02405.png}
			\end{figure}
		\end{frame}

		\begin{frame}
			\begin{figure}
				\centering
				\includegraphics[scale=0.4]{images/re-1000-512-04758.png}
			\end{figure}
		\end{frame}

		\begin{frame}
			\begin{figure}
				\centering
				\includegraphics[scale=0.4]{images/re-1000-512-06482.png}
			\end{figure}
		\end{frame}

		\begin{frame}
			\begin{figure}
				\centering
				\includegraphics[scale=0.4]{images/re-1000-512-07612.png}
			\end{figure}
		\end{frame}

		\begin{frame}
			\begin{figure}
				\centering
				\includegraphics[scale=0.4]{images/re-1000-512-08419.png}
			\end{figure}
		\end{frame}

		\begin{frame}
			\begin{figure}
				\centering
				\includegraphics[scale=0.4]{images/re-1000-512-10164.png}
			\end{figure}
		\end{frame}

		\begin{frame}
			\begin{figure}
				\centering
				\includegraphics[scale=0.4]{images/re-1000-512-11408.png}
			\end{figure}
		\end{frame}

		\begin{frame}
			\begin{figure}
				\centering
				\includegraphics[scale=0.4]{images/re-1000-512-12475.png}
			\end{figure}
		\end{frame}

	% section ergebnisse (end)

	\section{Zusammenfassung} % (fold)
	\label{sec:zusammenfassung}

		\begin{frame}
			\textbf{grundsätzliches Verfahren:}
			\begin{itemize}[label=$\circ$]
				\item problemabhängige spezialisierte Navier-Stokes-Gleichungen
				\item Diskretisierung durch Gitter und Finite-Differenzen-Methode
				\item Aufstellen des linearen Gleichungssystems
				\item Lösen durch Zeitschrittverfahren und Poisson-Löser
				\item Anzeigen der Lösung
			\end{itemize}
		\end{frame}

		\begin{frame}
			\textbf{Probleme:}
			\begin{itemize}[label=$\circ$]
				\item numerische Instabilität für auftretende Turbulenzen
				\item viele numerische Verfahren sind problemabhängig
				\item sehr hoher Aufwand für komplexe Geometrien
				\item meistens nur qualitativer Vergleich mit Experimenten möglich
			\end{itemize}

			\begin{figure}
				\includegraphics[scale = 0.2]{images/re-1000-64.png}
			\end{figure}
		\end{frame}

	% section zusammenfassung (end)
	
	\begin{frame}
		\frametitle{Referenzen}

		\begin{itemize}[label=$\circ$]
			\item Griebel und Dornseifer und Neunhoeffer, \textit{Numerical Simulation in Fluid Dynamics - A Practical Introduction}, 1998
			\item Ferziger und Peric, \textit{Computational Methods for Fluid Dynamics}, korrigierte 2.Auflage, 1997
			\item Durst, \textit{Grundlagen der Strömungsmechanik - Eine Einführung in die Theorie der Strömungen von Fluiden}, 2006
			\item Kincaid and Cheney, \textit{Numerical Analysis: Mathematics of Scientific Computing}, 3.Auflage, 2002
			\item Ansorg, Skript zu \textit{Thermodynamik und statistische Physik}, 2015/16
			\item \url{https://en.wikipedia.org/wiki/Computational_fluid_dynamics}
			\item \url{https://de.wikipedia.org/wiki/Navier-Stokes-Gleichungen}
		\end{itemize}
	\end{frame}

\end{document}